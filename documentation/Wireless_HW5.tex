\documentclass{article}
\usepackage[utf8]{inputenc}
\usepackage{amsmath, amssymb, amsthm, graphicx, float}
\allowdisplaybreaks

\title{Wireless Communication HW5}
\author{Arjun Jauhari(aj526)}
\date{8 November 2015}

\begin{document}

\maketitle

\section*{Problem 1}

SNR of OFDM system:\\
\begin{flalign*}
    &y_w = \lambda_w x_w + n_w ; w = 0,1,...,W-1 &\\
    &\lambda_w = \frac{1}{\sqrt{W}} \sum_{l=0}^{L-1} h_l exp\left(\frac{-j2\pi}{W}lw\right) &\\
    &h_l \sim \mathcal{CN} (0,1) ; n_w \sim \mathcal{CN} (0,N_0) ; W = 64 \\
    &SNR = \frac{\mathbb{E} \left[\|\mathbf{\lambda x}\|^2\right]}{\mathbb{E} \left[\|\mathbf{n}\|^2\right]} & \\
\end{flalign*}

\begin{flalign*}
    \mathbb{E} \left[\|\mathbf{\lambda x}\|^2\right] &= \mathbb{E} \left[\sum_{w=0}^{W-1}|\lambda_w x_w|^2\right] &\\
                                                     &= \sum_{w=0}^{W-1} \mathbb{E} \left[|\lambda_w x_w|^2\right] &\\
                                                     &= \sum_{w=0}^{W-1} \mathbb{E} \left[|\lambda_w|^2\right] \mathbb{E} \left[|x_w|^2\right]&\\
                                                     &= E_s \sum_{w=0}^{W-1} \mathbb{E} \left[|\lambda_w|^2\right]&\\
                                                     &= E_s \sum_{w=0}^{W-1} \mathbb{E} \left[|\frac{1}{\sqrt{W}} \sum_{l=0}^{L-1} h_l exp\left(\frac{-j2\pi}{W}lw\right)|^2\right]&\\
                                                     &= E_s \frac{1}{W} \sum_{w=0}^{W-1} \mathbb{E} \left[\sum_{l=0}^{L-1} |h_l|^2\right]&\\
                                                     &= E_s \frac{1}{W} \sum_{w=0}^{W-1} L &\\
                                                     &= E_s L &\\
\end{flalign*}

\begin{flushleft}
$\mathbb{E} \left[\|\mathbf{n}\|^2\right] = W N_0 $ \\
\end{flushleft}

\begin{flushleft}
Therefore,
$SNR = \frac{L E_s}{W N_0} $ \\
\end{flushleft}

\begin{figure}[H]
\centering
\includegraphics[width=10cm]{HW5_Q1.eps}
\caption{Problem 1}
\label{fig:prob1}
\end{figure}

\begin{flushleft}
    Part 1: From plot it can be seen that Diversity is 1 for simple OFDM case. \\
    Part 2: The error rate performance of OFDM with repitition coding is 2-3dB better than simple OFDM. \\
    Part 3: There is considerable performance gain in OFDM with interleaved repetition coding. \\
\end{flushleft}

\section*{Problem 2}
$v \sim \mathcal{CN} (m,\sigma^2)$

Given random variables $X_1, ..., X_n$  each with variance $\sigma^2$, $Var(\bar{X}_n) = \frac{\sigma^2}{n}$.

\begin{proof}
\begin{equation*}
\begin{split}
Var(\bar{X}_n) &= Var\left(\frac{X_1, ..., X_n}{n}\right) \\
&= (\frac{1}{n})^2 \sum_{i=1}^{n} \sigma_i^2 \\
&= \frac{\sigma^2}{n} \\
\end{split}
\end{equation*}
\end{proof}

\section{Integrals}
\begin{center}
\setlength{\tabcolsep}{1cm}
\def\arraystretch{2.5}
\begin{tabular}{c|c}
Integral & Evaluates to... \\
\hline
$\displaystyle \int_{a}^{b} k dx$ & $kx+c$
\end{tabular}
\end{center}

\section{Matrices}
\subsection{ The identity matrix}
\begin{itemize}
\item $ \mathbf{I}_{2 \times 2} = \left[
    \begin{array}{cc}
     1 & 0 \\
     0 & 1
    \end{array} \right] $
\item Symmetric and idempotent
\end{itemize}
\subsection{Matrix programming in MATLAB}
\begin{verbatim}
    B = [7, 8; 9, 10];
    
    % Find eigenvalues
    eB = eig(B)
\end{verbatim}


\end{document}

