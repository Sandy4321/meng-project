\documentclass{article}
\usepackage[utf8]{inputenc}
\usepackage[margin=1.2in]{geometry}
\usepackage{amsmath, amssymb, amsthm, graphicx, float}
\usepackage{subfig, subfloat}
\allowdisplaybreaks

\begin{document}
\section{Approach}
To address above mentioned issues a statistical approach is considered 
where first step is to identify the features which are relevant for the model.
Secondly, model needs to learn the parameters which govern the interaction
between these features.
Below I describe the model.
\subsection{Subscripts}
\begin{itemize}
    \item $i$ is subscript for question.
    \item $j$ is subscript for answer.
    \item $k$ is subscript for user.
\end{itemize}
\subsection{Notations}
\begin{enumerate}
    \item $u_k$: Quality measure of the $k$th user.
    \item $q_i$: Quality measure of the $i$th question.
    \item $va_{ij}$: Normalized votes corresponding to $j$th answer of $i$th question. \\
        Calculated as: $va_{ij} = \frac{|sa_{ij}|}{\sum_{j} |sa_{ij}|}$ \\
        where $sa_{ij}$ is the actual votes(score) read from data dump.
    \item $a_{ijk}$: Quality measure of $a_{ij}$th answer given by the $k$th user.
    \item $f_{acc}^{ij}$: Boolean flag telling if this answer was Accepted, read from data dump.
    \item $r_k$: Reputation of the $k$th user, read from data dump.
    \item $N_{a}^{i}$: Number of answer to $i$th question, read from data dump.
    \item $vq_{i}$: Number of votes to $i$th question, read from data dump.
\end{enumerate}

\subsection{Equations}
Below equations model the relation/dependence between
the above defined parameters. \\
Bold values are known features.
All the features were scaled between 0 \& 1 \\
\begin{enumerate}
    \item $a_{ijk} = f_a(u_k, \mathbf{va_{ij}}, \mathbf{f_{acc}^{ij}})$ \\
        $a_{ijk} = 1/3*u_k + 1/3*\mathbf{va_{ij}} + 1/3*\mathbf{f_{acc}^{ij}}$
    \item $u_k = f_u(\{a_{ijk}\}_{ij}, \{q_i\}_k, \mathbf{r_k})$ , where $\{a_{ijk}\}_{ij}$
        is set of all answers by user k\\
        $u_k = 1/3*mean\{a_{ijk}\}_{ij} + 1/3*mean\{q_i\}_k + 1/3*\mathbf{r_k}$
    \item $q_i = f_q(u_k, \mathbf{N_{a}^{i}}, \mathbf{vq_i}, \mathbf{\sum_{j} |sa_{ij}|})$, where $\sum_{j} |sa_{ij}|$
        is sum of votes for all the answers of $i$th question \\
        $q_i = 1/4*u_k + 1/4*\mathbf{N_{a}^{i}}+ 1/4* \mathbf{vq_i} + 1/4*\mathbf{\sum_{j} |sa_{ij}|}$
\end{enumerate}

\section{Status}
Currently, the model is very simple with $va_{ij} = \frac{|sa_{ij}|}{\sum_{j} |sa_{ij}|}$, where $sa_{ij}$ is the actual votes(score) read from data dump.\\
and $a_{ijk} = w_1*u_k + w_2*q_i + w_3*va_{ij} + w_4*f_{acc}^{ij}$ \\
\\
As of now weights $w_1, w_2, w_4$ are 0 and $w_3$ is 1, so $a_{ijk} = va_{ij}$ \\
\\
User quality is being modeled as: $u_k \sim \mathcal{N} (mean(\{a_{ijk}\}_{ij}), Var(\{a_{ijk}\}_{ij}))$ \\
\\
Question quality is still not modeled.

\end{document}

